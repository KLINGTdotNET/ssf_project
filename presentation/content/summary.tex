\section{Schlussbetrachtung}
\subsection{Alternative zur Template-Engine}
\begin{frame}{Erweiterungs- und Verbesserungsmöglichkeiten}
    \begin{block}{Alternative zur Template-Engine}
        \begin{itemize}
            \item Implementierung eines Sprachenmodells welches die Konstrukte der Zielsprache abbilden kann (Expressions, Statements, Conditions, Loops, \ldots)
            \item Sprachenmodell würde zu erzeugenden Code als \emph{Abstract Syntax Tree} enthalten
            \item Sprachenmodell würde Semantik und ein Rendermodul die Syntax kapseln
            \item Vorteile:
            \begin{itemize}
                \item Erweiterung um zusätzliche Zielsprachen durch Implementierung weiterer Rendermodules
                \item Formatierung des erzeugten Codes über Parameter änderbar (Einrückungstiefe, Klammerpositionen, \ldots)
                \item Optimierung des zu erzeugenden Codes durch Analyse des \textsc{Ast}
            \end{itemize}
        \end{itemize}
    \end{block}
\end{frame}

\subsection{Generatorsystem}
\begin{frame}
    \begin{block}{WADL}
        \begin{itemize}
            \item Generieren einer gesamten Client-Bibliothek für einen \textsc{Rest}ful Webservice
            \item Nutzung einer kompletten maschinenlesbare Beschreibung des Webservice als \textsc{Wadl} (Web Application Description Language) in Verbindung mit \textsc{Xsd}
            \item Generatorsystem bestehend aus:
            \begin{itemize}
                \item Codegenerator der Bibliothek für Zugriff auf Webserviceressourcen generiert
                \item Generator zur Erstellung der Datenklassen aus \textsc{Xsd}\footnote{analog dem hier vorgestellten Generator}
            \end{itemize}
        \end{itemize}
    \end{block}
\end{frame}

\subsection{Fazit}
\begin{frame}{Fazit}
    \begin{itemize}
        \item bildet derzeit nicht alle in \textsc{Xsd} erlaubten Regeln ab (sehr komplexe Typdefinitionen möglich, Anonyme Typen, \ldots)
        \item mit überschaubaren Änderungen produktiv einsetzbar
        \item Vertiefung der Kenntnisse im Umgang mit Template-Engines und XML-Bibliotheken in Python3
    \end{itemize}
\end{frame}