\section{Codegenerator}
\begin{frame}{Codegenerator}
    \begin{itemize}
        \item Generator\footnote{Codegenerator u. Generator wird hier Synonym verwendet} ist in Python 3.4 implementiert
        \item Zielsprache ist Java
        \item Überführung der \textsc{Xml} Schemabeschreibung in internes Datenmodell welches einer Template-Engine als Eingabe dient
        \item Änderung der Zielsprache über Anpassung der Templates
    \end{itemize}
\end{frame}

\begin{frame}{Generatorablauf}
    \centering
    \resizebox{!}{0.85\textheight}{
        \xdefinecolor{nearlywhite}{rgb}{0.85, 0.85, 0.85}

\tikzstyle{box} = [
    rectangle,
    rounded corners,
    %minimum width=3cm,
    minimum height=10mm,
    font=\sffamily\bfseries,
    text centered,
    draw=black,
    fill=nearlywhite
]

\tikzstyle{files} = [
    rectangle,
    rounded corners,
    %minimum width=3cm,
    minimum height=10mm,
    font=\sffamily\bfseries,
    text centered,
    draw=black,
    fill=nearlywhite
]

\tikzstyle{arrow} = [
    thick,
    ->,
    shorten <=1mm,
    shorten >=1mm,
    >=stealth   % arrowhead
]

\tikzstyle{note} = [
    font=\sffamily\small
]

\begin{tikzpicture}[node distance=10mm]

\node (xsd) [files, double copy shadow] {\textsc{Xsd}};
\node (parser) [box, below = of xsd] {\textsc{Xml}-Parser};
\node (schemamapper) [box, below = of parser] {Schemamapper};
\node (classmapper) [box, below = of schemamapper] {Classmapper};
\node (renderer) [box, below = of classmapper] {Renderer};
\node (javaRenderer) [box, below = of renderer] {Java-Renderer};

\draw [arrow] (xsd) -- (parser);
\draw [arrow] (parser) -- node[note, anchor=west] {XML-Tree} (schemamapper);
\draw [arrow] (schemamapper) -- node[note, anchor=west] {Schema-Model} (classmapper);
\draw [arrow] (classmapper) -- node[note, anchor=west] {Class-Model} (renderer);
\draw [arrow] (renderer) -- node[note, anchor=west] {Class-Model} node[note, anchor=east] {\enquote{Java}} (javaRenderer);

\end{tikzpicture}
    }
    %\caption{Ablauf des Codegenerators}
    \label{fig:flow}
\end{frame}

\subsection{Template-Engine}
\begin{frame}[fragile]{Template-Engine}
    \begin{itemize}
        \item Eine Template-Engine ist ein Textersetzungssystem welches \enquote{Templates} (Vorlagen) verarbeitet und darin enthaltene Platzhalter durch andere Inhalte ersetzt
        \item Vom Generator wird die \href{http://www.makotemplates.org/}{Mako} Template-Engine verwendet
    \end{itemize}
    \begin{lstlisting}[language=Java, caption=Beispieltemplate]
public class ${classname} {
    ${field.modifier} ${field.type} ${field.name}
        ${'= '+field.value if field.value else ''};
    ...
}
    \end{lstlisting}
\end{frame}
