\section{Codegenerator}
\begin{frame}{Codegenerator}
    \begin{itemize}
        \item Generator\footnote{Codegenerator u. Generator wird hier Synonym verwendet} ist in Python 3.4 implementiert
        \item Zielsprache ist Java
        \item Überführung der \textsc{Xml} Schemabeschreibung in internes Datenmodell welches einer Template-Engine als Eingabe dient
        \item Änderung der Zielsprache über Anpassung der Templates
    \end{itemize}
\end{frame}

\begin{frame}{Generatorablauf}
    \centering
    \resizebox{!}{0.85\textheight}{
        \xdefinecolor{nearlywhite}{rgb}{0.85, 0.85, 0.85}

\tikzstyle{box} = [
    rectangle,
    rounded corners,
    %minimum width=3cm,
    minimum height=10mm,
    font=\sffamily\bfseries,
    text centered,
    draw=black,
    fill=nearlywhite
]

\tikzstyle{files} = [
    rectangle,
    rounded corners,
    %minimum width=3cm,
    minimum height=10mm,
    font=\sffamily\bfseries,
    text centered,
    draw=black,
    fill=nearlywhite
]

\tikzstyle{arrow} = [
    thick,
    ->,
    shorten <=1mm,
    shorten >=1mm,
    >=stealth   % arrowhead
]

\tikzstyle{note} = [
    font=\sffamily\small
]

\begin{tikzpicture}[node distance=10mm]

\node (xsd) [files, double copy shadow] {\textsc{Xsd}};
\node (parser) [box, below = of xsd] {\textsc{Xml}-Parser};
\node (schemamapper) [box, below = of parser] {Schemamapper};
\node (classmapper) [box, below = of schemamapper] {Classmapper};
\node (renderer) [box, below = of classmapper] {Renderer};
\node (javaRenderer) [box, below = of renderer] {Java-Renderer};

\draw [arrow] (xsd) -- (parser);
\draw [arrow] (parser) -- node[note, anchor=west] {XML-Tree} (schemamapper);
\draw [arrow] (schemamapper) -- node[note, anchor=west] {Schema-Model} (classmapper);
\draw [arrow] (classmapper) -- node[note, anchor=west] {Class-Model} (renderer);
\draw [arrow] (renderer) -- node[note, anchor=west] {Class-Model} node[note, anchor=east] {\enquote{Java}} (javaRenderer);

\end{tikzpicture}
    }
    %\caption{Ablauf des Codegenerators}
    \label{fig:flow}
\end{frame}

\subsection{Datenmodelle}
\begin{frame}{Datenmodelle}
    \begin{block}{Schemamodell}
        \begin{itemize}
            \item kapselt die Spezifikation, in diesem Fall die Regeln aus der \textsc{Xsd}-Datei
            \item XML-Parser\footnote{\href{http://lxml.de/}{lxml}} erzeugt Objektbaum und validiert \textsc{Xsd}
            \item Über Objektbaum iterierern und wesentliche Informationen in Schemamodell übernehmen (\textsc{Xml} spezifische Inhalte weglassen)
            \item Schemamodell enthält Attribut-, Element- und Typdefinitionen sowie Namensraumangaben
        \end{itemize}
    \end{block}
\end{frame}

\begin{frame}{Datenmodelle}
    \begin{block}{Klassenmodell}
        \begin{itemize}
            \item Bildet Definitionen und Regeln des Schemamodells auf Konstrukte der Zielsprache ab
            %\item aus zeitlichen Gründen werden nicht alle im \textsc{Xsd} definierten Regeln abgebildet (können sehr komplex sein)
            \item Logik in Templates vermeiden, verringern die Wartbarkeit
            \item enthält Abhängigkeiten zwischen den Definitionen (Importieren von Klassen siehe \cref{lst:javaExample})
            \item enthält Serialisierungsmethoden\footnote{Deserialisierer aus Zeitgründen nicht mit generiert}
        \end{itemize}
    \end{block}
\end{frame}

\subsection{Template-Engine}
\begin{frame}[fragile]{Template-Engine}
    \begin{itemize}
        \item Eine Template-Engine ist ein Textersetzungssystem welches \enquote{Templates} (Vorlagen) verarbeitet und darin enthaltene Platzhalter durch andere Inhalte ersetzt
        \item Vom Generator wird die \href{http://www.makotemplates.org/}{Mako} Template-Engine verwendet
        \item Einhalten von Namenskonventionen (camelCase) durch Formatierungsmethoden
        \item Ordnerstruktur des generierten Java Packages wird aus der Namensraumangabe generiert, bspw. \texttt{http://api.facebook.com/1.0/} wird zu:\\
\small\texttt{\%ausgabepfad\%/.api.facebook.com/1\_0/...\\...src/main/java/com/facebook/api/}
    \end{itemize}
    \begin{lstlisting}[language=Java, caption=Beispieltemplate]
public class ${classname} {
    ${field.modifier} ${field.type} ${field.name}
        ${'= '+field.value if field.value else ''};
    ...
}
    \end{lstlisting}
\end{frame}

\begin{frame}[fragile]{Beispiel für generierte Klasse}
    \begin{adjustbox}{height=0.4\textheight,keepaspectratio}
        \begin{lstlisting}[language=Java, label=lst:javaExample, caption=Beispiel für eine generierte Java-Datei (gekürzt)]
package com.facebook.api;

import com.facebook.api.Uid;
import com.facebook.api.Vid;
import com.facebook.api.Time;

class VideoTag  {
    private Vid vid;
    private Uid subject;
...
    public void setVid(Vid vid) {
        this.vid = vid;
    }
    public Vid getVid() {
        return this.vid;
    }
...
    public Time getUpdatedTime() {
        return this.updated_time;
    }
    public String toXML() {
        return this.vid.toXML() + this.subject.toXML()
            + this.created_time.toXML() + this.updated_time.toXML();
    }
}
        \end{lstlisting}
    \end{adjustbox}
\end{frame}