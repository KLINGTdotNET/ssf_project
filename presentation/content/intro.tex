\section{Einführung}

\begin{frame}{Idee}
    \begin{itemize}
        \item Codegenerator der aus einer abstrakten Datenformatbeschreibung ein Java Package erzeugt\footnote{Implementierung existiert bereits als Teil von \href{https://jaxb.java.net/}{JAXB}}
        \item Datenformatbeschreibung als \emph{XML Schema Description}
        \item Ändern der Zielsprache sollte mit vertretbarem Aufwand möglich sein
    \end{itemize}
\end{frame}

\begin{frame}[squeeze]{Motivation}
    \begin{block}{Probleme}
        \begin{itemize}
            \item Daten müssen für die Kommunikation mit Webservices serialisiert werden
            \item in der Regel wird \textsc{Xml} unterstützt
            \item Struktur der erwarteten Daten kann als \emph{XML Schema Description} maschinenlesbar definiert werden
            \item \emph{Mapping} zwischen XML- und Klassendarstellung in der gewählten OO-Sprache um Daten einfach verarbeiten zu können
        \end{itemize}
    \end{block}
        \begin{block}{Generatorlösung}
        \begin{itemize}
            \item Generierung der Klassendarstellung aus \textsc{Xsd} in der gewünschten OO-Sprache (in diesem Fall Java)
            \item Erzeugen von Serialisierung- und Deserialisierungsmethoden\footnote{Werden derzeit nicht generiert.}
        \end{itemize}
        \end{block}
\end{frame}

\subsection{XSD}
\begin{frame}{\textsc{Xml} Schema Description \textendash{} \textsc{Xsd}}

\end{frame}

\begin{frame}[fragile]{\textsc{Xsd} einfache Typen}
    \begin{lstlisting}[language=XML, label=lst:simple, caption=Beispiel für einen einfachen Schematyp aus \cite{facebookXSD}]
<xsd:element name="auth_createToken_response"
    type="auth_token" />

<xsd:simpleType name="auth_token">
    <xsd:restriction base="xsd:string" />
</xsd:simpleType>

<!-- Beispiel für eine Instanz des Typs -->
<auth_createToken_response>foobar</auth_createToken_response>
    \end{lstlisting}
\end{frame}

\begin{frame}[fragile]{\textsc{Xsd} strukturierte Typen}
    \begin{lstlisting}[language=XML, label=lst:complex, caption=Beispiel für einen strukturierten Schematyp aus \cite{facebookXSD}]
<xsd:element name="video_getUploadLimits_response"
    type="video_limits" />

<xsd:complexType name="video_limits">
    <xsd:sequence>
        <xsd:element name="length" type="xsd:int" />
        <xsd:element name="size" type="xsd:long" />
    </xsd:sequence>
</xsd:complexType>

<!-- Beispiel für eine Instanz des Typs -->
<video_getUploadLimits_response>
    <length>21</length>
    <size>42</size>
</video_getUploadLimits_response>
    \end{lstlisting}
\end{frame}