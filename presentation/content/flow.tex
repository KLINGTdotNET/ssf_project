\xdefinecolor{nearlywhite}{rgb}{0.85, 0.85, 0.85}

\tikzstyle{box} = [
    rectangle,
    rounded corners,
    minimum width=30mm,
    minimum height=10mm,
    font=\sffamily\bfseries,
    text centered,
    draw=black,
    fill=nearlywhite
]

\tikzstyle{files} = [
    rectangle,
    rounded corners,
    minimum height=10mm,
    font=\sffamily\bfseries,
    text centered,
    draw=black,
    fill=nearlywhite
]

\tikzstyle{textNode} = [
    font=\sffamily,
    text centered,
    align=right
]

\tikzstyle{arrow} = [
    thick,
    ->,
    shorten <=1mm,
    shorten >=1mm,
    >=stealth   % arrowhead
]

\tikzstyle{note} = [
    font=\sffamily\small
]

\begin{tikzpicture}[node distance=10mm]

\node (xsd) [files, double copy shadow] {\textsc{Xsd}};
\node (parser) [box, below = of xsd] {\textsc{Xml}-Parser};
\node (parser_note) [textNode, left = of parser] {Einlesen und validieren\\der Spezifikation (XSD)};
\node (schemamapper) [box, below = of parser] {Schemamapper};
\node (schemamapper_note) [textNode, left = of schemamapper] {Überführen der wesentlichen Informationen\\der Spezif. in Schemamodell};
\node (classmapper) [box, below = of schemamapper] {Classmapper};
\node (classmapper_note) [textNode, left = of classmapper] {Klassenmodell (Eingabe der Template-Engine)\\mit Inhalten des Schemamodells befüllen\\Serialisierer erstellen \& Abhängigkeiten analysieren};
\node (renderer) [box, below = of classmapper] {Renderer};
\node (renderer_note) [textNode, left = of renderer] {Rendern des Quellcodes durch Template-Engine};
\node (javaRenderer) [box, below = of renderer] {Java-Renderer};

\draw [arrow] (xsd) -- (parser);
\draw [arrow] (parser) -- node[note, anchor=west] {XML-Tree} (schemamapper);
\draw [arrow] (schemamapper) -- node[note, anchor=west] {Schema-Model} (classmapper);
\draw [arrow] (classmapper) -- node[note, anchor=west] {Class-Model} (renderer);
\draw [arrow] (renderer) -- node[note, anchor=west] {Class-Model} node[note, anchor=east] {\enquote{Java}} (javaRenderer);

\end{tikzpicture}