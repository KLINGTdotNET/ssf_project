\section{Codegenerator}

Im Allgemeinen wird der Begriff \enquote{Generator} für verschiedene Technologien verwendet, u.a. Compiler u. Präprozessoren, Template-Metaprogramming in C++ und natürlich Codegeneratoren.

\myQuote{Ein Codegenerator ist ein Programm, welches aus einer höhersprachigen Spezifikation\footnote{mit anderen Worten: auf einem höheren Abstraktionslevel als das der zur Implementierung verwendetenProgrammiersprache}, einer Software oder eines Teilaspektes die Implementierung erzeugt.}{nach \cite{eisenecker}}

Der Codegenerierungsprozess erfolgt durch die folgenden Schritte:

\begin{enumerate}
    \item Einlesen der Spezifikation (\gls{XSD})
    \item Überführen der wesentlichen Informationen der Spezifikation in ein Schemamodell
    \item Erstellung eines Klassenmodells aus dem Datenmodell des Schemas
    \item Rendern des Quellcodes durch die Template-Engine mit dem Klassenmodell als Eingabe
\end{enumerate}

\subsection{Datenmodell}

Das Datenmodell beinhaltet die Informationen der Spezifikation und bildet somit die Basis für den Codegenerator.
Vom Generator werden zwei Datenmodelle zur Erzeugung des Quellcodes verwendet, das Schemamodell (\cref{sec:schemamodel}) und das Klassenmodell (\cref{sec:classmodel}).

\subsubsection{Schemamodell}
\label{sec:schemamodel}

Das Schemamodell kapselt die in der Spezifikation enthaltenen Informationen, in diesem Fall die Regeln aus der \gls{XSD}-Datei, welche die Struktur der XML-Daten festlegen.

\subsubsection{Klassenmodell}
\label{sec:classmodel}

Durch das Klassenmodell werden die Informationen aus dem Schemamodell auf die Konstrukte der Zielsprache abgebildet.

Serialisierer/Deserialisierer

\subsection{Architektur}

\subsection{Template-Engine}

\subsection{Alternative zum templatebasierten Ansatz}

Alternativ zur Verwendung einer \gls{template-engine} ist die Erstellung eines Modells für die Zielsprache des zu erzeugenden Codes möglich. Dieses \emph{Sprachenmodell} müsste dabei die Konstrukte der Zielsprache enthalten und den zu erzeugenden Code in Form eines \gls{AST} beinhalten.
Um aus dem Syntaxbaum Quellcode erzeugen können ist zusätzlich ein \emph{Renderer} zu implementieren. Gegenüber der Verwendung von Templates bietet dieser Ansatz folgende Vorteile:

\begin{itemize}
    \item Formatierung des erzeugten Codes über Parameter des \emph{Renderers}, bspw. Einrückungstiefe, Position von Klammern (auf neuer Zeile), \ldots
    \item Optimierung des zu erzeugenden Codes durch Analyse des \gls{AST}
\end{itemize}

Ein Nachteil ist der erhöhte Implementierungsaufwand.

\section{Installation und Verwendung}

Um den Generator nutzen zu können müssen noch die nötigen Abhängigkeiten installiert werden. Diese sind in der Datei \texttt{requirements.txt} im Stammverzeichnis des Projektes festgehalten und können mit Hilfe des Python Pakettools \emph{pip} wie folgt installiert werden: \texttt{pip install -r requirements.txt}.

Es werden zwei Kommandozeilenparameter vom Codegenerator erwartet, erstens eine \gls{URL} welche den Ort der \gls{XSD}-Datei angibt und zweitens den lokalen Pfad an dem das generierte Java Package angelegt werden sollen. Weiterhin existieren noch optionale Parameter zur Ausgabe von Debuginformationen (\texttt{-d/--debug}) oder\textemdash{}für die spätere Erweiterung gedacht\textemdash{}die Angabe der zu verwendenden Zielsprache mittels \texttt{-l/--lang} (Standard ist Java).

Die Hilfe kann mittels \texttt{-h/--help} ausgegeben werden, bei falscher Verwendung wird diese sowie ein Nutzungshinweis aber automatisch angezeigt.