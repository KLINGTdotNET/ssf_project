\section{Codegenerator}

\myQuote{Ein Codegenerator ist ein Programm, welches aus einer höhersprachigen Spezifikation\footnote{mit anderen Worten: auf einem höheren Abstraktionslevel als das der zur Implementierung verwendetenProgrammiersprache}, einer Software oder eines Teilaspektes die Implementierung erzeugt.}{nach \cite[S. 333]{eisenecker}}

Im Allgemeinen wird der Begriff \enquote{Generator} für verschiedene Technologien verwendet, u.a. Compiler u. Präprozessoren, Template-Metaprogramming in C++ und natürlich Codegeneratoren.

Der hier beschriebene Generator erzeugt ausführbaren Code, durch die Überführung einer XML Schemabeschreibung in ein internes Datenmodell welches als Eingabe für eine Template-Engine dient (siehe \cref{sec:template} auf \pageref{sec:template}).

Der Codegenerierungsprozess erfolgt dabei durch die folgenden Schritte:

\begin{enumerate}
    \item Einlesen und validieren der Spezifikation (\gls{XSD})
    \item Überführen der wesentlichen Informationen der Spezifikation in ein Schemamodell
    \item Befüllen des Eingabemodells des Generators (hier als Klassenmodell bezeichnet) mit den Spezifikationen aus dem Schemamodell
    \item Analysieren von Abhängigkeiten innerhalb des Klassenmodells
    \item Rendern des Quellcodes durch die Template-Engine
\end{enumerate}

\subsection{Datenmodell}

Das Datenmodell beinhaltet die Informationen der Spezifikation und bildet somit die Basis für den Codegenerator.
Vom Generator werden zwei Datenmodelle zur Erzeugung des Quellcodes verwendet, dass Schemamodell (\cref{sec:schemamodel}) und das Klassenmodell (\cref{sec:classmodel}).

\subsubsection{Schemamodell}
\label{sec:schemamodel}

Das Schemamodell kapselt die in der Spezifikation enthaltenen Informationen, in diesem Fall die Regeln aus der \gls{XSD}-Datei. Im ersten Schritt wird durch einen XML-Parser aus dem \gls{XSD} ein Objektbaum erzeugt. Gleichzeitig wird durch den Parser (lxml \cite{lxml}) geprüft ob die Spezifikation selbst wohlgeformtes \gls{XML} ist.
Der vom Parser erzeugte Baum enthält noch viele \gls{XML} spezifische Informationen welche im folgenden \emph{mapping} Schritt wegfallen. Das \texttt{schemamapper} Modul des Codegenerators iteriert hierfür über den Objektbaum und extrahiert die relevanten Informationen, wie bspw. Element- und Typdefinitionen welche darauffolgend in das Schemamodell eingefügt werden. Das Schemamodell selbst ist ein Python \emph{Dictionary}\footnote{auch als \emph{assoziatives Array} o. \emph{Map} bezeichnet} welches Angaben über die definierten Namensräume sowie die Attribut-, Element- u. Typdefinitionen enthält.

\subsubsection{Klassenmodell}
\label{sec:classmodel}

Um möglichst wenig Logik innerhalb der \emph{Templates} (siehe \cref{sec:template} S. \pageref{sec:template}) zu haben wird durch das Modul \texttt{classmapper} aus dem Schemamodell das Klassenmodell erzeugt. Dieses bildet die Definitionen und Regeln des Schemamodells auf Konstrukte der Zielsprache ab. Hierbei muss erwähnt werden das die XML Schemabeschreibung sehr komplexe Regeln zur Einschränkung von Wertebereichen und Definition von Typen erlaubt. Darunter fallen anonyme Typdefinitionen sowie Auftrittsreihenfolgen u. -häufigkeiten für Elemente \cite{XMLschema}. Dieses Regeln sind im Schemamodell enthalten, werden aber im Klassenmodell nicht übernommen da deren Abbildung den zeitlichen Rahmen diese Projektes übersteigen würde.

Durch das \texttt{classmapper} Modul werden die Element- und Typdefinitionen auf Abhängigkeiten untersucht und diese als \texttt{dependencies} Dictionary dem jeweiligen Klassenmodell hinzugefügt. Da aus jeder Element- und Typdefinition eine eigene Java-Klasse generiert wird müssen darin verwendete Typen, die in einer anderen Klasse-/Datei definiert sind, importiert werden. Diese Importanweisungen werden aus den \texttt{dependencies} generiert (siehe Zeile 3-5 in \cref{lst:javaExample}, S. \pageref{lst:javaExample}).
Außerdem werden referenzierte Typen aufgelöst und durch deren Definition ersetzt. Zusätzlich werden vom \texttt{classmapper} auch Serialiserungsmethoden erstellt, die aus der erzeugten Java-Klasse wieder \gls{XML} generieren.

\subsection{Architektur}

\subsection{Template-Engine}
\label{sec:template}

Eine Template-Engine ist ein Textersetzungssystem was Templates (Vorlagen) verarbeitet und darin enthaltene Platzhalter durch andere Inhalte ersetzt. Das \texttt{javaRenderer} Modul dieses Codegenerators verwendet dabei die \emph{Mako Template-Engine} \cite{mako}. Als Eingabe dient dabei das Klassenmodell mit welchem die Templatedateien für Element- bzw. Typdefinitionen befüllt werden. \cref{lst:javaExample} zeigt Java-Klasse welche aus der Typdefinition von \cref{lst:xsdVideoTag} generiert wurde. Das \texttt{javaRenderer} Modul übergibt der Template-Engine Methoden zur Formatierung von Klassen- und Methodennamen um bestimmte Namenskonventionen einzuhalten. In diesem Fall Methoden zur Erzeugung von \emph{camel-cased} Namen für Klassen und Methodennamen. Dies ist für die Funktionsweise nicht notwendigt, macht den erzeugten Code aber lesbarer.

\begin{lstlisting}[language=XML, label=lst:xsdVideoTag, caption=...]
<xsd:complexType name="video_tag">
  <xsd:sequence>
    <xsd:element name="vid" type="vid" />
    <xsd:element name="subject" type="uid" />
    <xsd:element name="created_time" type="time" />
    <xsd:element name="updated_time" type="time" />
  </xsd:sequence>
</xsd:complexType>
\end{lstlisting}

\begin{lstlisting}[language=Java, label=lst:javaExample, caption=Beispiel für eine generierte Java-Datei]
package com.facebook.api;

import com.facebook.api.Uid;
import com.facebook.api.Vid;
import com.facebook.api.Time;

class VideoTag  {

    private Vid vid;
    private Uid subject;
    private Time created_time;
    private Time updated_time;


    public void setVid(Vid vid) {
        this.vid = vid;
    }

    public Vid getVid() {
        return this.vid;
    }
    ...
    public Time getUpdatedTime() {
        return this.updated_time;
    }

    public String toXML() {
        return this.vid.toXML() + this.subject.toXML()
            + this.created_time.toXML()
            + this.updated_time.toXML();
    }
}
\end{lstlisting}

\subsection{Alternative zum templatebasierten Ansatz}

Alternativ zur Verwendung einer \gls{template-engine} ist die Erstellung eines Modells für die Zielsprache des zu erzeugenden Codes möglich. Dieses \emph{Sprachenmodell} müsste dabei die Konstrukte der Zielsprache enthalten und den zu erzeugenden Code in Form eines \gls{AST} beinhalten.
Um aus dem Syntaxbaum Quellcode erzeugen können ist zusätzlich ein \emph{Renderer} zu implementieren. Gegenüber der Verwendung von Templates bietet dieser Ansatz folgende Vorteile:

\begin{itemize}
    \item Formatierung des erzeugten Codes über Parameter des \emph{Renderers}, bspw. Einrückungstiefe, Position von Klammern (auf neuer Zeile), \ldots
    \item Optimierung des zu erzeugenden Codes durch Analyse des \gls{AST}
\end{itemize}

Ein Nachteil ist der erhöhte Implementierungsaufwand.

\section{Installation und Verwendung}

Um den Generator nutzen zu können müssen noch die nötigen Abhängigkeiten installiert werden. Diese sind in der Datei \texttt{requirements.txt} im Stammverzeichnis des Projektes festgehalten und können mit Hilfe des Python Pakettools \emph{pip} wie folgt installiert werden: \texttt{pip install -r requirements.txt}.

Es werden zwei Kommandozeilenparameter vom Codegenerator erwartet, erstens eine \gls{URL} welche den Ort der \gls{XSD}-Datei angibt und zweitens den lokalen Pfad an dem das generierte Java Package angelegt werden sollen. Weiterhin existieren noch optionale Parameter zur Ausgabe von Debuginformationen (\texttt{-d/--debug}) oder\textemdash{}für die spätere Erweiterung gedacht\textemdash{}die Angabe der zu verwendenden Zielsprache mittels \texttt{-l/--lang} (Standard ist Java).

Die Hilfe kann mittels \texttt{-h/--help} ausgegeben werden, bei falscher Verwendung wird diese sowie ein Nutzungshinweis aber automatisch angezeigt.