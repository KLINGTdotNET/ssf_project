\newglossaryentry{REST}{
    name=\textsc{Rest},
    description={
        \emph{Representational State Transfer} (deutsch: \enquote{Gegenständlicher Zustandstransfer}) ist ein Softwarearchitekturstil für Webanwendungen, welcher von Roy Fielding in seiner Dissertation \cite{fieldingDissertation} beschrieben wurde. Die Daten liegen dabei in eindeutig addressierbaren \emph{resources} vor. Die Interaktion basiert auf dem Austausch von \emph{representations} -- also ein Dokument was den aktuellen oder gewünschten Zustand einer resource beschreibt.
        Beispiel-URL für das Item \emph{84} aus dem Warenkorb \emph{42}:\\
        \texttt{http://foo.bar.de/baskets/84/item/42}
    }
}

\newglossaryentry{RESTful}{
    name=\textsc{Rest}ful,
    description={
        Als \emph{RESTful} bezeichnet man einen Webservice der den Prinzipien von REST entspricht
    },
    see=REST
}

\newglossaryentry{XML}{
    name=\textsc{Xml},
    description={
        Die \emph{Extensible Markup Language}, kurz \textsc{Xml}, ist eine Auszeichnungssprache (\enquote{Markup Language}), die eine Menge von Regeln beschreibt um Dokumente in einem mensch- und maschinenlesbaren Format zu kodieren \cite{XML10Specification}
    }
}

\newglossaryentry{XSD}{
    name=\textsc{Xsd},
    description={
        \emph{XML Schema Description}, auch nur \emph{XML Schema} ist eine Schemabeschreibungssprache und enthält Regeln für den Aufbau und zum Validieren einer XML-Datei. Die Beschreibung ist selbst wieder eine gültige XML-Datei
    },
    see=XML
}

\newglossaryentry{URI}{
    name=\textsc{Uri},
    description={
        Ein \enquote{Uniform Resource Identifier} (\gls{URI}) ist eine kompakte Zeichenkette zur Identifizierung einer abstrakten oder physischen Ressource. \ldots{} Eine Ressource ist alles was identifizierbar ist, beispielsweise elektronische Dokumente, Bilder, Dienste und Sammlungen von Ressourcen. (eigene Übersetzung von \cite{w3cURI}).
        %\emph{Unified Resource Identifier} ist ein Folge von Zeichen, die einen Name oder eine Web-Ressource identifiziert
    },
    plural=\textsc{Uri}s
}

\newglossaryentry{URL}{
    name=\textsc{Url},
    description={
        %\emph{Unified Resource Locator} sind eine Untermenge der \emph{URIs}. Der Unterschied besteht in der expliziten Angabe des Zugrissmechanismus und des Ortes (\enquote{Location}) durch \emph{URLs}, beispielsweise \texttt{http} oder \texttt{ftp}
        Der Begriff \enquote{Uniform Resource Locator} (\gls{URL}) bezieht sich auf eine Teilmenge von \glspl{URI}. \glspl{URL} identifizieren Ressourcen über den Zugriffsmechanismus, anstelle des Namens oder anderer Attribute der Ressource.
        (eigene Übersetzung von \cite{w3cURI}).
    },
    plural=\textsc{Url}s,
    see=URI
}

\newglossaryentry{URN}{
    name=\textsc{Urn},
    description={
        Eine Teilmenge der \glspl{URI}, die sogenannten \enquote{Uniform Resource Names} (\glspl{URN}), sind global eindeutige und beständige Bezeichner für Ressourcen. Sie müssen verfügbar bleiben auch wenn die bezeichnete Ressource nicht mehr erreichbar oder vorhanden ist. \ldots{} Der Unterschied zu einer \gls{URL} besteht darin, das ihr primärer Zweck in der dauerhaften Auszeichnung einer Ressource mit einem Bezeichner besteht.
        (eigene Übersetzung von \cite{w3cURI}).
    },
    plural=\textsc{Urn}s,
    see=URI
}

\newglossaryentry{WADL}{
    name=\textsc{Wadl},
    description={
        \emph{Web Application Description Language} ist eine maschinenlesbare Beschreibung einer HTTP-basierten Webanwendung
    },
    see=XML
}

\newglossaryentry{Python}{
    name=Python,
    description={
        \emph{Python} ist eine objekt-orientierte zu Bytecode kompilierte \emph{General Purpose Language}. Die Sprache kann durch C und C++ Module erweitert werden. Es existieren neben der Standartimplementierung \emph{Cpython} noch u.a. \emph{IronPython} (kompiliert zur \emph{Common Language Runtime} von .NET), \emph{Jython} (läuft auf der Java Virtual Machine) sowie der \emph{Just-In-Time Compiler} Pypy. Eine ähnliche Programmiersprache ist Ruby
    }
}

\newglossaryentry{template-engine}{
    name=Template-Engine,
    description={
        Eine \emph{Template-Engine} ersetzt markierte Bereiche in einer Template-Datei (i. Allg. Textdateien) nach vorgegebenen Regeln
    }
}

\newglossaryentry{AST}{
    name=Abstract Syntax Tree,
    description={
        Ein \emph{Abstrakter Syntaxbaum} ist die Baumdarstellung einer abstrakten Syntaktischen Struktur von Quellcode einer Programmiersprache. Jeder Knoten des Baumes kennzeichnet ein Konstrukt des Quellcodes. Der \emph{AST} stellt für gewöhnlich nicht alle Details des Quelltextes dar, beispielsweise formatierende Element wie etwa Klammern werden häufig weggelassen
    }
}