\documentclass[a4paper]{scrartcl}
%
% XeTeX specific
%
\usepackage{xltxtra}
\usepackage{fontspec}
\usepackage{xunicode}

% math
\usepackage{amsmath}
\usepackage{amsfonts}
\usepackage{amssymb}
\usepackage{amsthm}

% other
\usepackage{polyglossia}
\setdefaultlanguage{german}
\setotherlanguage{english}
\usepackage{csquotes}
\usepackage{graphicx}

\usepackage{color}

\usepackage{hyperref}
\hypersetup{
            citecolor=cyan,
            linkcolor=cyan,
            urlcolor=cyan,
            filecolor=cyan
            }

\hypersetup{hidelinks}

\usepackage{listings}
\setmonofont{Consolas}
\lstset{
        basicstyle=\small\ttfamily,
        numbers=left,
        numbersep=5pt,
        numberstyle=\tiny\color{blue}
}

\usepackage[german]{cleveref}
\usepackage{booktabs}

\usepackage[
    backend=biber,
    style=alphabetic-verb
    ]{biblatex}
\bibliography{main}

\usepackage[toc]{glossaries}
\makeglossaries

%
% Style options
%

\setsansfont{Linux Biolinum}
\setmonofont{Droid Sans Mono}

\newcommand{\documentTitle}{Some Title}
\newcommand{\university}{Universität Leipzig}
\newcommand{\documentAuthor}{Andreas Linz}
\newcommand{\documentAuthorShort}{A. Linz}
\newcommand{\documentSubtitle}{I am a subtitle.}
%\newcommand{\shortTitle}{}
%\newcommand{\universityDepartment}{}
%\newcommand{\documentKeywords}{}

\author{\documentAuthor}
\date{\today}

\begin{document}
    \begin{titlepage}
    \begin{center}
        {\Huge
            \textbf{\documentTitle{}}\\
        }
        \vspace{6pt}
        {\Large
        \textit{\documentSubtitle{}}
        }
        \vspace{12pt}
        \hrule
        \vspace{32pt}
        {
            \Large
            \textbf{\documentAuthor{}}
        }
        \\
        \vspace{32pt}
        \includegraphics[width=5cm]{resources/siegel_schwarz}
        \\
        \vspace{6pt}
        \sffamily{
            \Large
            \university
        }
        \vfill
        \today
    \end{center}
\end{titlepage}
    \clearpage

    \tableofcontents
    \clearpage

    \section{Einführung}

    Dieses Dokument beschreibt den für die Vorlesung \enquote{Softwaresystemfamilien} erstellten Codegenerator, welcher aus einer abstrakten Beschreibung von Datenformaten in Form einer \gls{XML} Schemabeschreibung (nachfolgend \gls{XSD}) ein Java Package erzeugt.

    \subsection{Motivation}

    Um Daten mit einem Webservices (z.B. eines \gls{RESTful} Webservice) austauschen zu können, müssen diese serialisiert werden, in der Regel wird dabei XML als Format für die serialisierten Daten unterstützt. Die Struktur der serialisierten Daten, die von dem Dienst empfangen bzw. an diesen gesendet werden können, kann in Form einer XML Schema Description maschinenlesbar definiert werden.

    Für eine effektive Verarbeitung der Daten in einer objektorientierten Programmiersprache (bspw. Java), benötigt man eine Abbildung aus der Beschreibung der XML-Darstellung und der Klassendarstellung in der gewählten Programmiersprache. Die manuelle Erstellung dieses \emph{Mappings} ist einerseits monoton und deshalb fehlerträchtig, sowie andererseits sehr zeitaufwendig, sollte sich die API und die verwendeten Datenstrukturen des Webservice ändern.

    Der Generator übernimmt diese Arbeit und erzeugt aus der Beschreibung des Datenschemas automatisch die Klassendarstellung in der gewünschten Programmiersprache.

    Wirklich spannend wie dieses Konzept erst als Teilaspekt eines Generatorsystems, welches eine komplette Client-Bibliothek für einen RESTful Webservice generiert, falls dieser neben der abstrakten Beschreibung der erwarteten Datenformate auch eine solche Beschreibung der verfügbaren Ressourcen bereitstellt, bspw. als \gls{WADL}-Datei (siehe \cref{sec:erweiterung} S. \pageref{sec:erweiterung}).

    \subsection{Was ist XML Schema Description (XSD)}

    Die \emph{XML Schema Description} o.a. nur \emph{XML Schema} ist eine Schemabeschreibungssprache mit Regeln um den Aufbau einer XML-Datei zu beschreiben. Außerdem können bestehende XML-Daten gegen das Schema validiert werden.

    \begin{lstlisting}[language=XML, caption=Beispiel für einen einfachen Schematyp]
<?xml version="1.0" encoding="UTF-8" ?>
<xsd:schema xmlns:xsd="http://www.w3.org/2001/XMLSchema"
    targetNamespace="http://api.facebook.com/1.0/"
    xmlns="http://api.facebook.com/1.0/" elementFormDefault="qualified">

<xsd:simpleType name="FacebookApiErrorCode">
  <xsd:restriction base="xsd:int" />
</xsd:simpleType>
    \end{lstlisting}

    \subsection{Python}

    \section{Generator}

    \subsection{Architektur}

    \subsection{Template-Engine}

    \section{Fazit}

    \subsection{Erweiterungsmöglichkeiten}
    \label{sec:erweiterung}

    \clearpage
    \section{Appendix}

    % glossary
    \newglossaryentry{REST}{
    name=\textsc{Rest},
    description={
        \emph{Representational State Transfer} (deutsch: \enquote{Gegenständlicher Zustandstransfer}) ist ein Softwarearchitekturstil für Webanwendungen, welcher von Roy Fielding in seiner Dissertation \cite{fieldingDissertation} beschrieben wurde. Die Daten liegen dabei in eindeutig addressierbaren \emph{resources} vor. Die Interaktion basiert auf dem Austausch von \emph{representations} -- also ein Dokument was den aktuellen oder gewünschten Zustand einer resource beschreibt.
        Beispiel-URL für das Item \emph{84} aus dem Warenkorb \emph{42}:\\
        \texttt{http://api.spreadshirt.net/api/v1/baskets/84/item/42}
    }
}

\newglossaryentry{RESTful}{
    name=\textsc{Rest}ful,
    description={
        Als \emph{RESTful} bezeichnet man einen Webservice der den Prinzipien von REST entspricht
    },
    see=REST
}

\newglossaryentry{XML}{
    name=\textsc{Xml},
    description={
        Die \emph{Extensible Markup Language}, kurz \textsc{Xml}, ist eine Auszeichnungssprache (\enquote{Markup Language}), die eine Menge von Regeln beschreibt um Dokumente in einem mensch- und maschinenlesbaren Format zu kodieren \cite{XML10Specification}
    }
}

\newglossaryentry{XSD}{
    name=\textsc{Xsd},
    description={
        \emph{XML Schema Description}, auch nur \emph{XML Schema} ist eine Schemabeschreibungssprache und enthält Regeln für den Aufbau und zum Validieren einer XML-Datei. Die Beschreibung ist selbst wieder eine gültige XML-Datei
    },
    see=XML
}

\newglossaryentry{URI}{
    name=\textsc{Uri},
    description={
        Ein \enquote{Uniform Resource Identifier} (\gls{URI}) ist eine kompakte Zeichenkette zur Identifizierung einer abstrakten oder physischen Ressource. \ldots{} Eine Ressource ist alles was identifizierbar ist, beispielsweise elektronische Dokumente, Bilder, Dienste und Sammlungen von Ressourcen. (eigene Übersetzung von \cite{w3cURI}).
        %\emph{Unified Resource Identifier} ist ein Folge von Zeichen, die einen Name oder eine Web-Ressource identifiziert
    },
    plural=\textsc{Uri}s
}

\newglossaryentry{URL}{
    name=\textsc{Url},
    description={
        %\emph{Unified Resource Locator} sind eine Untermenge der \emph{URIs}. Der Unterschied besteht in der expliziten Angabe des Zugrissmechanismus und des Ortes (\enquote{Location}) durch \emph{URLs}, beispielsweise \texttt{http} oder \texttt{ftp}
        Der Begriff \enquote{Uniform Resource Locator} (\gls{URL}) bezieht sich auf eine Teilmenge von \glspl{URI}. \glspl{URL} identifizieren Ressourcen über den Zugriffsmechanismus, anstelle des Namens oder anderer Attribute der Ressource.
        (eigene Übersetzung von \cite{w3cURI}).
    },
    plural=\textsc{Url}s,
    see=URI
}

\newglossaryentry{URN}{
    name=\textsc{Urn},
    description={
        Eine Teilmenge der \glspl{URI}, die sogenannten \enquote{Uniform Resource Names} (\glspl{URN}), sind global eindeutige und beständige Bezeichner für Ressourcen. Sie müssen verfügbar bleiben auch wenn die bezeichnete Ressource nicht mehr erreichbar oder vorhanden ist. \ldots{} Der Unterschied zu einer \gls{URL} besteht darin, das ihr primärer Zweck in der dauerhaften Auszeichnung einer Ressource mit einem Bezeichner besteht.
        (eigene Übersetzung von \cite{w3cURI}).
    },
    plural=\textsc{Urn}s,
    see=URI
}

\newglossaryentry{WADL}{
    name=\textsc{Wadl},
    description={
        \emph{Web Application Description Language} ist eine maschinenlesbare Beschreibung einer HTTP-basierten Webanwendung
    },
    see=XML
}

\newglossaryentry{Python}{
    name=Python,
    description={
        \emph{Python} ist eine objekt-orientierte zu Bytecode kompilierte \emph{General Purpose Language}. Die Sprache kann durch C und C++ Module erweitert werden. Es existieren neben der Standartimplementierung \emph{Cpython} noch u.a. \emph{IronPython} (kompiliert zur \emph{Common Language Runtime} von .NET), \emph{Jython} (läuft auf der Java Virtual Machine) sowie der \emph{Just-In-Time Compiler} Pypy. Eine ähnliche Programmiersprache ist Ruby
    }
}

\newglossaryentry{template-engine}{
    name=Template-Engine,
    description={
        Eine \emph{Template-Engine} ersetzt markierte Bereiche in einer Template-Datei (i. Allg. Textdateien) nach vorgegebenen Regeln
    }
}

\newglossaryentry{AST}{
    name=Abstract Syntax Tree,
    description={
        Ein \emph{Abstrakter Syntaxbaum} ist die Baumdarstellung einer abstrakten Syntaktischen Struktur von Quellcode einer Programmiersprache. Jeder Knoten des Baumes kennzeichnet ein Konstrukt des Quellcodes. Der \emph{AST} stellt für gewöhnlich nicht alle Details des Quelltextes dar, beispielsweise formatierende Element wie etwa Klammern werden häufig weggelassen
    }
}
    \printglossary[type=main,title={Glossar},toctitle={Glossar}]

    % bibliography
    \addcontentsline{toc}{section}{Bibliographie}
    \printbibliography
\end{document}
