\section{Fazit}
\label{sec:conclusion}

Das Projekt diente hauptsächlich dazu sich mit dem Thema Codegenerierung in Python zu beschäftigen und Erfahrung mit Template-Engines und XML Bibliotheken in dieser Sprache zu sammeln. Der entstandene Generator kann auch mit überschaubaren Änderungen so erweitert werden das man ihn produktiv einsetzen kann.

Wie in \cref{sec:motivation} (S. \pageref{sec:motivation}) erwähnt ist der Einsatzzweck eines solchen Generators optimalerweise als Teil eines Generatorsystems zu sehen, welches aus einer kompletten Beschreibung eines \gls{RESTful} Webservice in Form von \gls{WADL}- und \gls{XSD}-Dateien eine Client-Bibliothek erzeugt. Mit einem solchen Generatorsystem kann man bei Änderungen an der Webservice-API eine neue Client-Bibliothek generieren die den Zugriff auf alle Ressourcen erlaubt ohne das der Nutzer sich mit Details der Kommunikation auseinandersetzen muss. % näher erläutern

\subsection{Erweiterungsmöglichkeiten}
\label{sec:erweiterung}

Um das erzeugte Java Paket sinnvoll nutzen zu können sollten noch Deserialisierungsmethoden generiert werden, die die Typklassen direkt aus der XML-Darstellung initialisieren. Zustätzlich sollten noch mehrere Aspekte der XML Schemadefinition bei der Generierung berücksichtigt werden, wie anonyme Typdefinitionen, die man als z.B. Innere Klassen generieren könnte.

\begin{lstlisting}[language=XML]
<xsd:element name="bar">
  <xsd:complexType>
    <xsd:sequence>
      <xsd:element name="foo" type="xsd:decimal"/>
    </xsd:sequence>
    <xsd:attribute name="baz" type="xsd:boolean" />
  </xsd:complexType>
</xsd:element>
\end{lstlisting}

\begin{lstlisting}[language=Java]
class Bar {
    class Anonym {
        private Double foo;
        private Boolean baz;
        ...
        public String toXML() {
            return "<foo>"+this.foo+"</foo>"
        }
    }

    private Anonym value;
    ...
    public String toXML() {
        "<bar " + "baz=" + value.baz + "/>" + value.toXML() + "</bar>"
    }
}
\end{lstlisting}

Die Implementierung zusätzlicher Zielsprachen durch die Erstellung weiterer Templates ist vorgesehen, ebenfalls ist ein Kommandozeilenparameter \texttt{-l/--lang LANGUAGE} bereits implementiert unter dessen Berücksichtigung das \texttt{render} Modul den entsprechenden Renderer auswählt.
